~\newline
 \tabulinesep=1mm
\begin{longtabu} spread 0pt [c]{*{2}{|X[-1]}|}
\hline
\rowcolor{\tableheadbgcolor}\textbf{ Projekt B\+P\+PC C\+Value Example }&\textbf{ 2017  }\\\cline{1-2}
\endfirsthead
\hline
\endfoot
\hline
\rowcolor{\tableheadbgcolor}\textbf{ Projekt B\+P\+PC C\+Value Example }&\textbf{ 2017  }\\\cline{1-2}
\endhead
Název projektu\+: &\hyperlink{namespace_c_value__bool}{C\+Value\+\_\+bool}, \hyperlink{namespace_c_value___t_week_day}{C\+Value\+\_\+\+T\+Week\+Day}, {\bfseries \hyperlink{namespace_c_value__long}{C\+Value\+\_\+long}}, {\bfseries \hyperlink{namespace_c_value___t_student}{C\+Value\+\_\+\+T\+Student}} \\\cline{1-2}
Autor 1\+: &{\ttfamily 186135}, {\bfseries Magáth} Marek, {\ttfamily \href{mailto:xmagat01@feec.vutbr.cz}{\tt xmagat01@feec.\+vutbr.\+cz}} \\\cline{1-2}
Autor 2\+: &{\ttfamily 186125}, {\bfseries Kuna} Ján, {\ttfamily \href{mailto:xkunaj01@feec.vutbr.cz}{\tt xkunaj01@feec.\+vutbr.\+cz}} \\\cline{1-2}
Autor 3\+: &{\ttfamily 173688}, {\bfseries Levrinc} Adam, {\ttfamily \href{mailto:xlevri01@feec.vutbr.cz}{\tt xlevri01@feec.\+vutbr.\+cz}} \\\cline{1-2}
Autor 4\+: &{\ttfamily ID}, {\bfseries Příjmení} Jméno, {\ttfamily login/email} \\\cline{1-2}
Datum zadání\+: &23.\+10.\+2017 \\\cline{1-2}
Datum finálního odevzdání\+: &13.\+12.\+2017 (včetně) \\\cline{1-2}
\end{longtabu}
~\newline
~\newline
 \subsection*{Úvodní poznámky}

Cílem tohoto testovacího projektu je pomoci Vám navrhnout a odzkoušet zadané varianty třídy {\ttfamily C\+Value}. Projekt již obsahuje plně funkční varianty \hyperlink{class_c_value__bool_1_1_c_value}{C\+Value\+\_\+bool\+::\+C\+Value} a \hyperlink{class_c_value___t_week_day_1_1_c_value}{C\+Value\+\_\+\+T\+Week\+Day\+::\+C\+Value}. Vaším úkolem je naprogramovat další dvě varianty třídy C\+Value dle zvoleného zadání a ověřit jejich fungování a kompatibilitu pomocí dodané funkce \hyperlink{main_8cpp_a0ddf1224851353fc92bfbff6f499fa97}{main()} v souboru \hyperlink{main_8cpp}{main.\+cpp}.

Vývoj projektu je možný v libovolném prostředí a kompilátoru C++ (\href{http://gcc.gnu.org/}{\tt g++}, \href{http://clang.llvm.org/}{\tt clang++}), ale referenčním překladačem bude \href{http://www.visualstudio.com/}{\tt Microsoft Visual Studio Professional 2017}.

\begin{DoxyAttention}{Attention}
Tento projekt (tj. adresář C\+Value) je určen pouze pro testování a slouží pro kontrolu funkčnosti nových typů {\ttfamily C\+Value}. Hodnocení funkčnosti nových tříd {\ttfamily C\+Value} proto proběhne v tomto projektu. Hodnocení zadání, komentářů a zdrojů datového kontejneru, ale proběhne na základě dat v adresáři {\bfseries Project} (tj. \char`\"{}modrá\char`\"{} varianta dokumentace vygenerovaná pomocí Doxygen).
\end{DoxyAttention}
~\newline
\subsection*{Doporučený postup}


\begin{DoxyEnumerate}
\item Pročtěte si tuto dokumentaci (\hyperlink{doc}{Dokumentace projektu C\+Value}) a dodané zdrojové texty. Význam symbolů a barevného značení u zobrazovaných grafů popisuje tato \href{graph_legend.html}{\tt legenda}.
\item Proveďte trasování dodané funkce pro obě varianty tříd {\ttfamily C\+Value}.
\item Promyslete smysl celého programu a význam odlišností v jednotlivých variantách tříd C\+Value.
\item Navrhněte a implementujte obě nové varianty tříd {\ttfamily C\+Value} do vlastních jmenných prostorů.
\item Ověřte jejich správnou funkci na {\bfseries nemodifikovaném} těle funkce {\ttfamily main}.
\item Doplňte dokumentaci Doxygen pro nové vytvořené soubory tříd {\ttfamily C\+Value} (preferovaným jazykem C++ zdrojových textů je angličtina, případně čeština bez diakritiky).
\item Jakmile Vámi navržené varianty tříd {\ttfamily C\+Value} budou fungovat správně a ohodnotí Vám je vyučující (společně s průběžným hodnocením hlavičkových souborů projektu), zkopírujte a začleňte si jejich zdrojové texty do dalšího projektu\+: C\+Hash\+Map a následně Project
\end{DoxyEnumerate}

~\newline
\subsection*{Poznámky k řešení\+:}

\begin{DoxyNote}{Note}
Na následujícím odkazu najdete \href{https://my.mindnode.com/s2Bbn2gFS8pHZFGcZJhRj4U7zmxx8ivygkoPCuZz}{\tt Myšlenkovou mapu k projektu}, kterou jsme vytvořili v úvodu přednášky věnované projektu. Pro lepší orientaci uvádíme i krátkou motivaci k pojmu třída na odkazu\+: \href{http://www.uamt.feec.vutbr.cz/~richter/vyuka/1314_ppc/bppc/cviceni/motivace_trida.html}{\tt Motivace třída}. ~\newline
 
\end{DoxyNote}
\begin{DoxyParagraph}{Id}
\hyperlink{_introduction_8md}{Introduction.\+md} 616 2017-\/11-\/27 15\+:32\+:14Z xkunaj01 
\end{DoxyParagraph}
