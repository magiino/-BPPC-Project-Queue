~\newline
 \tabulinesep=1mm
\begin{longtabu} spread 0pt [c]{*{2}{|X[-1]}|}
\hline
\rowcolor{\tableheadbgcolor}\textbf{ Projekt B\+P\+PC C\+Hash Map Example }&\textbf{ 2017  }\\\cline{1-2}
\endfirsthead
\hline
\endfoot
\hline
\rowcolor{\tableheadbgcolor}\textbf{ Projekt B\+P\+PC C\+Hash Map Example }&\textbf{ 2017  }\\\cline{1-2}
\endhead
Název projektu\+: &\hyperlink{class_c_hash_map}{C\+Hash\+Map} \\\cline{1-2}
Autor 1\+: &{\ttfamily 186135}, {\bfseries Magáth} Marek, {\ttfamily \href{mailto:xmagat01@feec.vutbr.cz}{\tt xmagat01@feec.\+vutbr.\+cz}} \\\cline{1-2}
Autor 2\+: &{\ttfamily 186125}, {\bfseries Kuna} Ján, {\ttfamily \href{mailto:xkunaj01@feec.vutbr.cz}{\tt xkunaj01@feec.\+vutbr.\+cz}} \\\cline{1-2}
Autor 3\+: &{\ttfamily 173688}, {\bfseries Levrinc} Adam, {\ttfamily \href{mailto:xlevri01@feec.vutbr.cz}{\tt xlevri01@feec.\+vutbr.\+cz}} \\\cline{1-2}
Autor 4\+: &{\ttfamily ID}, {\bfseries Příjmení} Jméno, {\ttfamily login/email} \\\cline{1-2}
Datum zadání\+: &23.\+10.\+2017 \\\cline{1-2}
Datum finálního odevzdání\+: &13.\+12.\+2017 (včetně) \\\cline{1-2}
\end{longtabu}
~\newline
~\newline
 \subsection*{Úvodní poznámky}

Cílem tohoto testovacího projektu je pomoci Vám navrhnout a odzkoušet třídu {\ttfamily \hyperlink{class_c_hash_map}{C\+Hash\+Map}} a pomocné třídy nutné pro její správnou činnost ({\ttfamily \hyperlink{struct_c_hash_map_1_1_t_item}{C\+Hash\+Map\+::\+T\+Item}}, {\ttfamily \hyperlink{class_c_pair}{C\+Pair}}). Projekt již obsahuje plně funkční varianty {\ttfamily \hyperlink{class_c_value__bool_1_1_c_value}{C\+Value\+\_\+bool\+::\+C\+Value}} a {\ttfamily \hyperlink{class_c_value___t_week_day_1_1_c_value}{C\+Value\+\_\+\+T\+Week\+Day\+::\+C\+Value}}. Dále obsahuje strukturu pro uzel vázaného seznamu {\ttfamily \hyperlink{struct_c_hash_map_1_1_t_item}{C\+Hash\+Map\+::\+T\+Item}}, která je základní součástí množiny prvků připadající každému indexu ve třídě {\ttfamily \hyperlink{class_c_hash_map}{C\+Hash\+Map}}. Třídy {\ttfamily \hyperlink{class_c_hash_map}{C\+Hash\+Map}} a {\ttfamily \hyperlink{class_c_pair}{C\+Pair}} jsou připravené, až na funkčnost vybraných metod, jejichž kód je nutné dopsat. Třída {\ttfamily \hyperlink{class_c_pair}{C\+Pair}} slouží jako datový člen uzlu vázaného seznamu {\ttfamily \hyperlink{struct_c_hash_map_1_1_t_item}{C\+Hash\+Map\+::\+T\+Item}} a obsahuje dvojici hodnot\+: klíč a hodnota. Třída {\ttfamily \hyperlink{class_c_hash_map}{C\+Hash\+Map}} slouží pro vlastní manipulaci s hodnotami (vkládání, výběr, třídění, ...) Vaším úkolem je naprogramovat (a pomocí kódu ve funkci main ověřit) chybějící kód metod tříd {\ttfamily \hyperlink{class_c_pair}{C\+Pair}} a {\ttfamily \hyperlink{class_c_hash_map}{C\+Hash\+Map}}. Kód musí fungovat pro všechny varianty třídy C\+Value dle zvoleného zadání.

Vývoj projektu je možný v libovolném prostředí a kompilátoru C++ (\href{http://gcc.gnu.org/}{\tt g++}, \href{http://clang.llvm.org/}{\tt clang++}), ale referenčním překladačem bude \href{http://www.visualstudio.com/}{\tt Microsoft Visual Studio Professional 2017}.

\begin{DoxyAttention}{Attention}
Tento projekt (tj. adresář \hyperlink{class_c_hash_map}{C\+Hash\+Map}) je určen pouze pro testování a slouží pro kontrolu funkčnosti doplněného kódu tříd {\ttfamily \hyperlink{class_c_pair}{C\+Pair}}, {\ttfamily \hyperlink{class_c_hash_map}{C\+Hash\+Map}} a nových typů {\ttfamily C\+Value}. Hodnocení funkčnosti tříd {\ttfamily \hyperlink{class_c_pair}{C\+Pair}} a {\ttfamily \hyperlink{class_c_hash_map}{C\+Hash\+Map}} proto proběhne v tomto projektu. Hodnocení zadání, komentářů a zdrojů datového kontejneru, ale proběhne na základě dat v adresáři {\bfseries Project} (tj. \char`\"{}modrá\char`\"{} varianta dokumentace vygenerovaná pomocí Doxygen).
\end{DoxyAttention}
~\newline
\subsection*{Doporučený postup}


\begin{DoxyEnumerate}
\item Pročtěte si tuto dokumentaci (\hyperlink{doc}{Dokumentace projektu C\+Hash\+Map}) a dodané zdrojové texty. Význam symbolů a barevného značení u zobrazovaných grafů popisuje tato \href{graph_legend.html}{\tt legenda}.
\item Proveďte trasování dodané funkce main.
\item Promyslete strukturu celého programu a význam tříd \hyperlink{class_c_pair}{C\+Pair}, \hyperlink{class_c_hash_map}{C\+Hash\+Map}, \hyperlink{struct_c_hash_map_1_1_t_item}{C\+Hash\+Map\+::\+T\+Item} a C\+Value.
\item Navrhněte a implementujte chybějící kód tak, aby byl výsledný kód funkční pro všechny varianty tříd {\ttfamily C\+Value}.
\item Ověřte správnou funkci nového kódu na {\bfseries nemodifikovaném} těle funkce {\ttfamily main}.
\item Doplňte dokumentaci Doxygen pro nově vytvořené metody (preferovaným jazykem C++ zdrojových textů je angličtina, případně čeština bez diakritiky).
\item Jakmile Vámi navržené varianty tříd budou fungovat správně a ohodnotí Vám je vyučující (společně s průběžným hodnocením hlavičkových souborů projektu), zkopírujte a začleňte si jejich zdrojové texty do Vašeho projektu\+: {\bfseries Project} (C\+Kontejner).
\end{DoxyEnumerate}

~\newline
\subsection*{Poznámky k řešení\+:}

\begin{DoxyRefDesc}{Todo}
\item[\hyperlink{todo__todo000001}{Todo}]Do tohoto textu nezapomeňte doplnit hlavičku (tj. jména řešitelů, varianty tříd {\ttfamily C\+Value}, datum zadání).\end{DoxyRefDesc}


\begin{DoxyNote}{Note}
Na následujícím odkazu najdete \href{https://my.mindnode.com/s2Bbn2gFS8pHZFGcZJhRj4U7zmxx8ivygkoPCuZz}{\tt Myšlenkovou mapu k projektu}, kterou jsme tvořili během přednášky věnované projektu a také \href{https://my.mindnode.com/hGXwS1Fwu9UbP7HD9E5GjsE8hyzBgtz2hPg2zWJQ}{\tt Mind mapu třídy C\+Hash\+Map}. Pro lepší orientaci uvádíme i krátkou motivaci k pojmu třída na odkazu\+: \href{http://www.uamt.feec.vutbr.cz/~richter/vyuka/1314_ppc/bppc/cviceni/motivace_trida.html}{\tt Motivace třída}. ~\newline
 
\end{DoxyNote}
\begin{DoxyParagraph}{Id}
\hyperlink{_introduction_8md}{Introduction.\+md} 190 2017-\/11-\/07 12\+:56\+:21Z xlevri01 
\end{DoxyParagraph}
