~\newline

\begin{DoxyEnumerate}
\item U všech tří projektů (C\+Value, \hyperlink{class_c_hash_map}{C\+Hash\+Map}, Project) vyřešit poznámky z minulé opravy. Projít stránku Bug a doxygen prikazy bug přepsat na bugfix. Za poznámku z opravy připsat, kdo opravil a jak opravil (tj. jak se poznámka vyřešila -\/ u jednoduchých poznámek nebo obecných upozornění stačí jméno, kdo upravil na bugfix). Výsledný projekt tedy bude obsahovat funkční C\+Value, \hyperlink{class_c_hash_map}{C\+Hash\+Map} a Project (shrnutý v Project, ale zpětně funkční).~\newline
~\newline

\item Zadání a realizaci projektu kontrolovat proti původní dodané šabloně zadání a doplnit chybějící metody a prvky. (= konstruktory a destruktor. Možnost zjistit ID a počty prvků třídy. Operátory včetně nečlenského a konverzního. Streamy. Privátní a veřejné metody. Dvojici metod \char`\"{}typu Reverzuj\char`\"{}.) Nejčastější chyby (zatím popsané pro datový typ loňský C\+X\+Item) najdete na -\/ \href{http://www.uamt.feec.vutbr.cz/%7Erichter/vyuka/XPPC/bppc/cviceni/chyby_hlavicka.html}{\tt http\+://www.\+uamt.\+feec.\+vutbr.\+cz/$\sim$richter/vyuka/\+X\+P\+P\+C/bppc/cviceni/chyby\+\_\+hlavicka.\+html}~\newline
~\newline

\item Projekt musí jít přeložit bez chyb a bez významných varování (warnings). Bude obsahovat knihovnu Check.~\newline
~\newline

\item Realizujte kontejner pro jeden svůj datový typ C\+Value, který musí projít testy prvního projektu C\+Value s originálním main. (tj. změna na C\+Value pro bool by neměla způsobit problémy v kontejneru).~\newline
~\newline

\item V hlavičkovém souboru v definici třídy budou pouze prototypy metod a funkcí. Celé metody s těly budou buď ve zdrojovém souboru nebo inline za definicí třídy.~\newline
~\newline

\item Napište main ve stejném stylu jako u projektu C\+Value a \hyperlink{class_c_hash_map}{C\+Hash\+Map} s voláním všech vytvořených metod kontejneru. Část kódu demonstrující vstup z klávesnice dejte do komentáře \char`\"{}//\char`\"{}.~\newline
~\newline

\item Kde to u parametrů nebo návratových hodnot je možné, bude použita reference. U parametrů které se nemění včetně {\ttfamily this} bude {\ttfamily const}.~\newline
~\newline

\item Vytvořte testy. Pro každou metodu bude test obsahovat alespoň tři testy (= testový {\ttfamily assert}. Alespoň jeden bude \char`\"{}významný\char`\"{} tj. nebudou např.\+  všechny jeho proměnné vytvořené implicitními kontejnery).~\newline
~\newline

\item U čtyř testů proveďte i test paměti. U ostatních může být pouze test na funkčnost.~\newline
~\newline

\item Vytvoření dokumentace pomocí doxygen proběhne bez varování (warnings) (zkuste na příkazovém řádku příkaz {\ttfamily \char`\"{}doxygen 2$>$err.\+txt\char`\"{}} , který uloží do souboru err.\+txt chybová hlášení a na obrazovce ponechá pouze informační texty).~\newline
~\newline

\item V dokumentaci bude ponecháno zadání z doby opravování -\/ nemazat.\+ K němu dopište změny na základě poznámek (bug) a změny, ke kterým jste se rozhodli během realizace.~\newline
~\newline

\item V dokumentaci popište, co je váš kontejner, definujte jeho chování vůči klíči a hodnotě (= popište koncepci třídy). \+Popište vlastnosti a chování metod charakteristických pro daný kontejner.~\newline
~\newline

\item Napište dokumentaci pro členy třídy.\+ U metod popište jejich činnost, funkci argumentů a návratové hodnoty. Dokumentace musí souhlasit s aktuální verzí zdrojových textů.~\newline
~\newline

\item Napište hodnocení/závěr, ve kterém popište nedostatky vašeho řešení (Co jste případně nestihli. Proč jste byli nuceni změnit zadáni, hlavičku (oproti odevzdané verzi). Jaký k tomu byl důvod.).~\newline
~\newline

\item {\bfseries Všichni členové teamu budou mít srovnatelný a dostatečný počet komitů do svn.} ~\newline
~\newline
\begin{DoxyParagraph}{Id}
\hyperlink{_tasks_8md}{Tasks.\+md} 1763 2017-\/12-\/11 15\+:25\+:59Z xkunaj01 
\end{DoxyParagraph}

\end{DoxyEnumerate}